\chapter{Conclusion}

During this stage, I had the occasion to work with CMake and Subversion, tools
that I never used before. I also learned XML basics and discovered that this
language is a powerful way to format data for exchange and storage. Although I
worked alone on my project, I could work in group because I used code done by
other people when I was working with the simulator. \\

Beside the things I learned, I considerably improved my Qt knowledges and
gained some knowledges in Linux (shell). Because the Institute is an
international centre, the common language to communicate was English so that I
could practice and improved my knowledges.\\

This stage allowed me to use some notions I learned during my studies. Thus, I
used the sockets for network programming, the Model-View-Controler (MVC)
concept (called Model-View in Qt) and the concurrent access protection (by using
a mutex). To implement all these notions, I could use exclusively Qt, keeping
the client application multi-platform.\\

On the professional side, I could realize what was really life in a company. I
had permanently in mind that my work had to be evolutive. Indeed, one of the
future modifications will be the display of the simulation in 3D on the client
window. This idea forced me to be as most logical as possible in my code
organization so that my successor would easily retake the project.\\

With this stage, I gained a good professional, but also personnal, experience
for my future carreer as computer scientist.
