\chapter{Client}

\section{Description}

The client application is multi-platform: it must be able to run under Windows,
Linux and Mac OS. The client allows the user to configure and run a simulation.
The client knows nothing: all information about available object types,... are
gathered from the server. 

\section{Client features}

\subsection{Modes}

The client has two modes:
\begin{itemize}
 \item basic : only basic options are displayed;
 \item advanced : all options are displayed.\\
\end{itemize}

Another difference between these two types is that, in advanced mode, the user
is able select the port number. Only non-reserved port numbers (from 49153 to
65535) are available. In basic mode, the number of default port is used.

\subsection{Launch and shutdown the server application}

The client application is able to launch and shutdown the server. The server is
launched by establishing an SSH connexion to the computer on which the server
will run (see \ref{sshAndQt} for more information). The server is shut down by
sending a request through the socket.\\

\subsection{Log window}

A log window, that can be hidden by the user, displays messages and errors
coming from the server and simulation results.

\section{Encountered problems and their solution}

\subsection{SSH and Qt}
\label{sshAndQt}

SSH (\textit{\textbf{S}ecure \textbf{Sh}ell}) allows to connect to a remote
machine using an encrypted connection. The user must have an account on the
remote machine (but not necessarily with the same username) to authenticate. The
user has two ways to get authenticated:
\begin{enumerate}
 \item By entering his password.
 \item By using public keys. Public keys use a passphrase instead of a
password. The advantage of this technique is that the authentication can be
automatic and SSH requires the passphrase only once.
\end{enumerate}

In most applications, the first technique will not work because the application
will not be able to provide correctly the password (this is a security of SSH).
This is why all SSH authentication in this project use public keys (that must
have been correctly configured before).

\subsection{Launching the server application through an SSH connection}

The first problem I had when launching the server application through an SSH
connection was the time to wait before the client can attempt to connect to the
server. The server application can take from less than one second to several
seconds to initialize its environment. \\

The server opens its socket after the environment initialization. Thus, the
first solution found was that the client attempts to connect to the server
every 100 ms. This solution looked good: while the server environment is
initialized, the connection is refused (this error is ignored by the client
during that process) and once the socket is open, the client gets connected and
stops its attempts.\\

By doing this, I found another problem: if another server instance is already
running on the selected port, the client gets connected to this instance
(instead of the instance that is initializing). This is not what the user
wants. The new server instance can not check if the requested port is free
before trying to open a socket on it (which is only done after environment
initilization). The solution found to solve this problem was simply to check if
there is no server application running on this port by executing a script on
the remote machine before lauching the server application.
