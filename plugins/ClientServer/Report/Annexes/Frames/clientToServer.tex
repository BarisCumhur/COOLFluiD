\chapter{From client to server}

\section{Get tree}
\label{getTree}

\subsection{Description}

This frame is used by the client to request the XML tree.

\subsection{Structure}

\begin{lstlisting}[language=XML]
<ClientServerXML>
  <getTree/>
</ClientServerXML>
\end{lstlisting}

%-------------------------------------------------------------
%-------------------------------------------------------------

\section{Add node}
\label{addNode}

\subsection{Description}

This frame is used by the client to add a node. 

\subsection{Structure}

\begin{lstlisting}[language=XML]
<ClientServerXML>
  <addNode path="/path/to/parent" name="nodeName" type="concreteType" 
     abstype="abstractType" />
</ClientServerXML>
\end{lstlisting}

\subsection{Arguments}
\descrXMLAttr{path}{Path from the root to the parent node of the new node.
Node names are separated by a \textit{slash} ('/').\\}
\descrXMLAttr{name}{Name of the new node.\\}
\descrXMLAttr{type}{Concrete type of the new node.\\}
\descrXMLAttr{absType}{Abstract type of the new node.}

\subsection{Server answer on success}

The server answers by sending the updated tree \seeAlso{tree}{tree} to all
clients.

\subsection{Possible errors}

\begin{itemize}
 \item The path is not valid.
 \item The name contains a space.
 \item A child node with this name already exists for this parent.
 \item The abstract type does not exist.
 \item The concrete does not exist or does not inherit from the given abstract
type.
\end{itemize}

%-------------------------------------------------------------
%-------------------------------------------------------------

\section{Rename node}
\label{renameNode}

\subsection{Description}

This frame is used by the client to rename a node. It is not considered as an
error if the new name is equal to the current node name.

\subsection{Structure}

\begin{lstlisting}[language=XML]
<ClientServerXML>
  <renameNode path="/path/to/node" name="newName" />
</ClientServerXML>
\end{lstlisting}

\subsection{Arguments}
\descrXMLAttr{path}{Path from the root to the node to rename.
Node names are separated by a \textit{slash} ('/').\\}
\descrXMLAttr{name}{New name for the node.}

\subsection{Server answer on success}

The server answers by sending the updated tree \seeAlso{tree}{tree} to all
clients.

\subsection{Possible errors}

\begin{itemize}
 \item The path is not valid.
 \item The name contains a space.
 \item A child node with this name already exists for this parent.
\end{itemize}

%-------------------------------------------------------------
%-------------------------------------------------------------

\section{Delete node}
\label{deleteNode}

\subsection{Description}

This frame is used by the client to delete a node.

\subsection{Structure}

\begin{lstlisting}[language=XML]
<ClientServerXML>
  <deleteNode path="/path/to/node" />
</ClientServerXML>
\end{lstlisting}

\subsection{Arguments}
\descrXMLAttr{path}{Path from the root to the node to delete.
Node names are separated by a \textit{slash} ('/').\\}

\subsection{Server answer on success}

The server answers by sending the updated tree \seeAlso{tree}{tree} to all
clients.

\subsection{Possible errors}

\begin{itemize}
 \item The path is not valid.
\end{itemize}

%-------------------------------------------------------------
%-------------------------------------------------------------

\section{Modify node}
\label{modifyNode}

\subsection{Description}

This frame is used by the client to modify a node, which means add or modify
options of a node.

\subsection{Structure}

\begin{lstlisting}[language=XML]
<ClientServerXML>
  <modifyNode>
    <modOptions path="/path/to/node">
      (modified options)
    </modOptions>
    <addOptions path="/path/to/node">
      (new options)
    </addOptions>
  </modifyNode>
</ClientServerXML>
\end{lstlisting}

\subsection{Arguments}
\descrXMLAttr{path}{Path from the root to the node to modify.
Node names are separated by a \textit{slash} ('/').\\}

\subsection{Data structure}

Data structure is composed of two distinct sections. The first section
(\texttt{modOptions}) contains all existing options to modify. The second
section (\texttt{addOptions}) contains all new options to add. Both sections
are facultative and \texttt{addOptions} can be set before \texttt{modOptions}.
The protocol allows to have several \texttt{addOptions} and
\texttt{modeOptions} sections so that it is possible to modify more than one
node at a time.

\subsection{Server answer on success}

The server answers by sending the updated tree \seeAlso{tree}{tree} to all
clients.

\subsection{Possible errors}

\begin{itemize}
 \item The path is not valid.
 \item An option in \texttt{modOptions} section does not exist.
 \item An option in \texttt{addOptions} section already exists.
\end{itemize}

%-------------------------------------------------------------
%-------------------------------------------------------------

\section{Get abstract types}
\label{getAbstractType}

\subsection{Description}

This frame is used by the client to request abstract types list for a specified
type.

\subsection{Structure}

\begin{lstlisting}[language=XML]
<ClientServerXML>
  <getAbstractTypes typeName="typeName" />
</ClientServerXML>
\end{lstlisting}

\subsection{Arguments}
\descrXMLAttr{typeName}{Type name.\\}

\subsection{Server answer on success}

The server answers by sending the list of the requested abstract types
\seeAlso{abstractTypes}{abstractTypes}.

\subsection{Possible errors}

\begin{itemize}
 \item The given type name does not exist.
\end{itemize}

%-------------------------------------------------------------
%-------------------------------------------------------------

\section{Get concrete types}
\label{getConcreteType}

\subsection{Description}

This frame is used by the client to request concrete types list for a specified
abstract type.

\subsection{Structure}

\begin{lstlisting}[language=XML]
<ClientServerXML>
  <getConcreteTypes typeName="abstractTypeName" />
</ClientServerXML>
\end{lstlisting}

\subsection{Arguments}
\descrXMLAttr{typeName}{Abstract type name.\\}

\subsection{Server answer on success}

The server answers by sending the list of the requested concrete types
\seeAlso{concreteTypes}{concreteTypes}.

\subsection{Possible errors}

\begin{itemize}
 \item The given type name does not exist.
\end{itemize}

%-------------------------------------------------------------
%-------------------------------------------------------------

\section{Open directory}
\label{openDir}

\subsection{Description}

This frame is used by the client to open a directory on the server, read it and
send its contents.

\subsection{Structure}

\begin{lstlisting}[language=XML]
<ClientServerXML>
  <openDir path="/path/to/dir"/>
</ClientServerXML>
\end{lstlisting}

\subsection{Arguments}
\descrXMLAttr{path}{Path to directory to open. This path can be absolute or
relative (from default path). If the path is empty, a default path is used.\\}

\subsection{Server answer on success}

The server answers by sending the directory contents
\see{dirContent}{dirContent}.

\subsection{Possible errors}

\begin{itemize}
 \item The path is not valid.
\end{itemize}

%-------------------------------------------------------------
%-------------------------------------------------------------

\section{Open file}
\label{openFile}

\subsection{Description}

This frame is used by the client to open a case file

\begin{lstlisting}[language=XML]
<ClientServerXML>
  <openFile filename="filename" />
</ClientServerXML>
\end{lstlisting}

\subsection{Arguments}
\descrXMLAttr{filename}{Filename (and its path) to open. The path
can be absolute or relative (from default path).\\}

\subsection{Server answer on success}

The server answers by sending a ACK frame \seeAlso{ack}{ack}.

\subsection{Possible errors}

\begin{itemize}
 \item The path is not valid.
\end{itemize}

%-------------------------------------------------------------
%-------------------------------------------------------------

\section{Shutdown server}
\label{shutdownServer}

\subsection{Description}

This frame is used by the client to shutdown the server

\subsection{Structure}

\begin{lstlisting}[language=XML]
<ClientServerXML>
  <shutdownServer />
</ClientServerXML>
\end{lstlisting}

\subsection{Server answer on success}

The server never answers to this frame.

%-------------------------------------------------------------
%-------------------------------------------------------------

\section{Run simulation}
\label{runSimulation}

\subsection{Description}

This frame is used by the client to launch the simulation

\subsection{Structure}

\begin{lstlisting}[language=XML]
<ClientServerXML>
  <runSimulation />
</ClientServerXML>
\end{lstlisting}

\subsection{Server answer on success}

The server answers by sending a ACK frame \seeAlso{ack}{ack}.

\subsection{Possible errors}

\begin{itemize}
 \item No case file open.
 \item A simulation is already running.
\end{itemize}
