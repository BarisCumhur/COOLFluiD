\documentclass[11pt]{article}
%*******************************************************************
%--Packages
%*******************************************************************
\usepackage[final]{graphicx,epsfig}
\usepackage{subfigure}
\usepackage{subfigmat}
\newcommand{\noi}{\noindent}

\setlength{\parindent}{0pt}
\setlength{\parskip}{5pt plus 2pt minus 1 pt}
\topmargin  -5mm
\evensidemargin 8mm
\oddsidemargin  2mm
\textwidth  158mm
\textheight 230mm
%\renewcommand{\baselinestretch}{1.0}
\frenchspacing
\sloppy

%%%%%%%%%%%%%%%%%%%%%%%%%%%%%%%%%%%%%%%%%%%%%%%%%%%%%%%%%%%%%%%%%%%%%%%

\begin{document}
\pagestyle{empty}

\begin{center}
  {\fontsize{14}{20}\bf 
    User-manual for COOLFluiD RDS-Unsteady Navier-Stokes \\
    (version 2011.04) \\[10pt]}
\end{center}

\begin{center}
  {Nad\`{e}ge Villedieu, \underline{villedie@vki.ac.be} \\
    Von Karman Institute, Aeronautics \& Aerospace Dept.}
\end{center}

The test case presented here is solving a viscous unsteady flow over a cylinder at $Re=140$ and $M=0.2$. It is set up with adimensionnalized values, which means that the diameter of the cylinder 
is one and the density and velocity are one. The other parameters are computed such that the Mach and the Reynolds numbers are as expected.

\section{Options that are common to all CFcases}

Necessary modules to run this test case:
\begin{verbatim}
Simulator.Modules.Libs = libForwardEuler libCFmeshFileWriter libCFmeshFileReader 
libTecplotWriter libNavierStokes libFluctSplit libFluctSplitSystem 
libFluctSplitSpaceTime libFluctSplitSpaceTimeNavierStokes libGambit2CFmesh
\end{verbatim}
{\bf WATCH OUT:} The order of the libraries is very important!!! It must be in this order.


\subsection{Physical Model}
The physical model should be set as follows:
\begin{verbatim}
Simulator.SubSystem.Default.PhysicalModelType = NavierStokes2D

Simulator.SubSystem.NavierStokes2D.DiffTerm.ViscosityLaw = Constant
Simulator.SubSystem.NavierStokes2D.DiffTerm.Constant.Value = 0.0000632911
\end{verbatim}
The viscosity coefficient is set to a constant value such that the Reynolds number is
respected.

\subsection{Pseudo-time solver}
In this case we use an explicit Forward Euler solver for the pseudo-time iterations. The pseudo-time process
is explained in \cite{AIAA-paper}.
To choose the type of pseudo-time discretisation the option is:
\begin{verbatim}
Simulator.SubSystem.ConvergenceMethod = FwdEuler
\end{verbatim}
And the CFL is defined by:
\begin{verbatim}
Simulator.SubSystem.FwdEuler.Data.CFL.Value = 0.6
\end{verbatim}
{\bf WATCH OUT:} This CFL number is not dictated by the past shield condition (\cite{mr-thesis}).\\

A possible set up of the stop condition of the pseudo-time is:
\begin{verbatim}
Simulator.SubSystem.FwdEuler.StopCondition = RelativeNormAndMaxIter
Simulator.SubSystem.FwdEuler.RelativeNormAndMaxIter.RelativeNorm = -5.0
Simulator.SubSystem.FwdEuler.RelativeNormAndMaxIter.MaxIter = 500
\end{verbatim}
In this case, the (pseudo-time) iterations will stop when the 500 iterations
has been executed, or when the relative norm of the residual is $-5.0$ (the relative norm is 
the difference between the residual of the first and of the last iteration). This option is the most convenient,
but it is also possible to check the absolute norm and the maximal number of iterations (\texttt{AbsoluteNormAndMaxIter}) or
only the maximal number of iterations (\texttt{MaxNumberSteps}).\\

It is possible choose to print or not the evolution of the residuals:
\begin{verbatim}
Simulator.SubSystem.FwdEuler.Data.PrintHistory = false
\end{verbatim}
The residuals can be computed using different norms, it is better
to use L2 norm. Then, it is also possible to define which variable should 
fufill the stop condition:
\begin{verbatim}
Simulator.SubSystem.FwdEuler.Data.NormRes = L2
Simulator.SubSystem.FwdEuler.Data.L2.MonitoredVarID = 0
Simulator.SubSystem.FwdEuler.Data.L2.ComputedVarID = 0 1 2 3
\end{verbatim}
\subsection{The mesh}
The mesh is loaded and setup as follow:
\begin{verbatim}
Simulator.SubSystem.Default.listTRS = InnerCells wall in out symm_0 symm_1
Simulator.SubSystem.MeshCreator = CFmeshFileReader
Simulator.SubSystem.CFmeshFileReader.Data.FileName = kvs2.CFmesh
\end{verbatim}

\section{RDS}
\subsection{Solver}
To choose the residual method as solver:
\begin{verbatim}
Simulator.SubSystem.SpaceMethod = FluctuationSplit
\end{verbatim}
When one want to restart from a previous solution this option should be on \texttt{true} 
\begin{verbatim}
Simulator.SubSystem.FluctuationSplit.Restart = true
\end{verbatim}
{\bf WATCH OUT:} The default value of this boolean is \texttt{false}


To define which variables are used in each step of the method (to write the solution, to update the 
residual, to average the solution in each cell in order to compute the Jacobian and to compute the diffusive term)
\begin{verbatim}
Simulator.SubSystem.FluctuationSplit.Data.SolutionVar  = Cons
Simulator.SubSystem.FluctuationSplit.Data.UpdateVar  = Cons
Simulator.SubSystem.FluctuationSplit.Data.DistribVar = Cons
Simulator.SubSystem.FluctuationSplit.Data.LinearVar  = Cons
Simulator.SubSystem.FluctuationSplit.Data.DiffusiveVar = Cons
\end{verbatim}
{\bf WATCH OUT:} \textttt{DiffusiveVar} should always be Cons when using \texttt{UnsteadyNavierStokes}. The
\texttt{LinearVar} are not very important when CRD \cite{mr-thesis} is used to compute the residual. It is better to put it to Roe when
shocks are considered and N or Nlim schemes are used

To setup the quadrature rule used to compute the residual on the sides of the element:
\begin{verbatim}
Simulator.SubSystem.FluctuationSplit.Data.IntegratorQuadrature = GaussLegendre
Simulator.SubSystem.FluctuationSplit.Data.IntegratorOrder = P3
\end{verbatim}
{\bf WATCH OUT:} Those line are extremely important for any system computations because the 
default quadrature rule is accurate only for linear functions

\subsection{Initialization}
It is necessary to intialize the field and also the boundaries 
where a strong condition is used.

\begin{verbatim}
Simulator.SubSystem.FluctuationSplit.InitComds = InitState \
 						 StrongNoSlipWallAdiabaticNS2D 


Simulator.SubSystem.FluctuationSplit.InitNames = InField InWall

Simulator.SubSystem.FluctuationSplit.InField.applyTRS = InnerCells
Simulator.SubSystem.FluctuationSplit.InField.Vars = x y
Simulator.SubSystem.FluctuationSplit.InField.Def =  1.0 1.0 0.0 350.9087378

Simulator.SubSystem.FluctuationSplit.InWall.applyTRS = wall
\end{verbatim}

\section{Unsteady RDS for Navier-Stokes}

\subsection{Writing solutions}
It is possible to change make the name of the output file dependent on the time:
\begin{verbatim}
Simulator.SubSystem.CFmesh.AppendTime = true
Simulator.SubSystem.Tecplot.AppendTime = true
\end{verbatim}

\subsection{Defenition of the time step}
There are two possibilities to determine the time step.
First, one can fix the value of the time step constant 
over all the computation:

\begin{verbatim}
Simulator.SubSystem.SubSystemStatus.TimeStep = 0.002
\end{verbatim}
{\bf WATCH OUT:} When running in parallel it is mandatory to use this method!!

The second option is to determine the time step at every time iteration,
then it is computed by $\Delta t= DT \_ Ratio \cdot MaxDT$, where $MaxDT$
denotes the maximum time step respecting the past shield condition \cite{mr-thesis}

\begin{verbatim}
Simulator.SubSystem.SubSystemStatus.TimeStep = 0.002
Simulator.SubSystem.SubSystemStatus.ComputeDT = MaxDT
Simulator.SubSystem.SubSystemStatus.MaxDT.DT_Ratio = 0.5
\end{verbatim}
{\bf WATCH OUT:} The first line is still necessary to fix the first time step. Moreover, this option can be used
only if the simulation is ran in serial. Indeed, up to now, the time step are not synchronized.

Since there is a restriction on the time step, it is usually interested to know
what is this limit. So, in general, it is better to first launch one simulation on one processor with two time steps
and with non fixed time step. Like that it is possible to know what time step should be used. Anyway, in the case of
a viscous flow, usually, the time step should be smaller that the past shield condition.

\subsection{Stop condition}
There are two options to stop the computation. The first one is to
decide to stop the simulation when a threshold time is reached: 
\begin{verbatim}
Simulator.SubSystem.StopCondition   = MaxTime
Simulator.SubSystem.MaxTime.maxTime = 500.0
\end{verbatim}
{\bf WATCH OUT:} The time that will be reached is not exactly the one asked. Indeed, the simulation will stop
as soon as the \texttt{maxTime} is reached or passed: at the last iteration,if the time step is bigger than the difference between the current 
time and the final one, then the final time will be bigger than \texttt{MaxTime}.


The other possibility is to stop the computation when a number of time step
has been reached
\begin{verbatim}
Simulator.SubSystem.StopCondition   = MaxNumberSteps
Simulator.SubSystem.MaxNumberSteps.nbSteps = 13000
\end{verbatim}



\subsection{Solver}

First, we choose what is the strategy used to discretize the unsteady problem.
The option used here is the space-time where the advective residual is computed using CRD (see \cite{mr-thesis}) 
\begin{verbatim}
Simulator.SubSystem.FluctuationSplit.Data.FluctSplitStrategy = STU_CRD	
\end{verbatim}
{\bf WATCH OUT:} Only this strategy has been validated for the unsteady Navier-Stokes

The distribution coefficients are defined by the splitter, here we use the space-time version 
of LDA, two versions are possible \texttt{STKT$\_$SysLDAC} and \texttt{STKS$\_$SysLDAC}, only the first
one has been validated for the current problem.
\begin{verbatim}
Simulator.SubSystem.FluctuationSplit.Data.SysSplitter = STKT_SysLDAC
\end{verbatim}
{\bf WATCH OUT:} When starting the simulation from scratch it may be better to start with the N scheme (\texttt{STKT$\_$SysNC}).\\

The command to define the diffusive term is:
\begin{verbatim}
Simulator.SubSystem.FluctuationSplit.Data.DiffusiveTerm = UnsteadyNavierStokes
\end{verbatim}


\subsection{Boundary conditions}
In this case the wall of the cylinder is a non-slipping wall with an adiabatic condition 
for the energy, the inlet and outlet are strong subsonic conditions which are non-reflecting, and finally, for 
the top and bottom boundaries we use an hybrid boundary which check if the flow is getting in or out and
then apply the corresponding conditions.
\begin{verbatim}
Simulator.SubSystem.FluctuationSplit.BcComds = StrongNoSlipWallAdiabaticNS2D \
                                               StrongSubOutletNonRefEuler2DCons \
                                               StrongSubInletNonRefEuler2DCons \
						WeakFarFieldEuler2DCons \
						WeakFarFieldEuler2DCons

Simulator.SubSystem.FluctuationSplit.BcNames = BCWall \
                                               BCOut \
                                               BCIn \
                                               BCSym1 \
                                               BCSym2
\end{verbatim}

In the case of the wall and the outlet, only the TRS where the condition should be applied is
necessary:
\begin{verbatim}
Simulator.SubSystem.FluctuationSplit.BCWall.applyTRS = wall
Simulator.SubSystem.FluctuationSplit.BCOut.applyTRS = out
\end{verbatim}

For the inlet it is necessary to define the inflow:
\begin{verbatim}
Simulator.SubSystem.FluctuationSplit.BCIn.applyTRS = in
Simulator.SubSystem.FluctuationSplit.BCIn.Vars = x y t
Simulator.SubSystem.FluctuationSplit.BCIn.InFlow = 1.0 1.0 0.0 350.9087378
\end{verbatim}


For the farfield it is necessary to give the total and static pressure and the total temperature.
\begin{verbatim}
Simulator.SubSystem.FluctuationSplit.BCSym1.applyTRS = symm_0
Simulator.SubSystem.FluctuationSplit.BCSym1.Ttot = 0.488793964
Simulator.SubSystem.FluctuationSplit.BCSym1.Ptot = 140.6641324
Simulator.SubSystem.FluctuationSplit.BCSym1.P = 140.1634951

Simulator.SubSystem.FluctuationSplit.BCSym2.applyTRS = symm_1
Simulator.SubSystem.FluctuationSplit.BCSym2.Ttot = 0.488793964
Simulator.SubSystem.FluctuationSplit.BCSym2.Ptot = 140.6641324
Simulator.SubSystem.FluctuationSplit.BCSym2.P = 140.1634951
\end{verbatim}


\subsection{Post-processing}
It is possible to write in a file the evolution of the solution in one point:
\begin{verbatim}
Simulator.SubSystem.DataPostProcessing = DataProcessing
Simulator.SubSystem.DataPostProcessingNames = DataProcessing2
Simulator.SubSystem.DataProcessing2.ProcessRate = 1
Simulator.SubSystem.DataProcessing2.Comds = SamplingPoint
Simulator.SubSystem.DataProcessing2.Names = SPoint
Simulator.SubSystem.DataProcessing2.SPoint.OutputFile = samplingbis.plt
Simulator.SubSystem.DataProcessing2.SPoint.location = 1.0 0.0
Simulator.SubSystem.DataProcessing2.SPoint.SaveRate = 1
\end{verbatim}



\begin{thebibliography}{02}
\bibitem{AIAA-paper} L. Koloszar ,N. Villedieu, T. Quintino, J. Anthoine and P. Rambaud. Application of Residual Distribution Method for Acoustic Wave Propagation.
{\emph{AIAA journal}}, AIAA 2009-3116 (2009)
\bibitem{mr-thesis} M. Ricchiuto. {\emph{Construction and Analysis of Compact Residual Discretizations for Conservation Laws on Unstructured Meshes}}. Universit{\'e} Libre de Bruxelles (2005),
ISBN: 2-930389-16-8
\end{thebibliography}

\end{document}
